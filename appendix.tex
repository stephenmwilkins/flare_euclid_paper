

\section{Identifying high-$z$ galaxies with Euclid}

With one visible and 3 near-infrared bands \euclid\ provides the ability to robustly identify galaxies at high-redshift $z>6$ via the Lyman-$\alpha$/limit break. To demonstrate \euclid's capabilities we run a simple simulation in which we generate synthetic photometry assuming a simple $\beta$ model. In this model the underlying rest-frame ultraviolet spectrum is described by a simple power-law: $L_{\nu}\propto \lambda^{\beta+2}$ with ISM/IGM abosorption described by MADAU96.

We then add noise assuming the \euclid\ point-source depths and use the {\sc eazy} \insref\ photometric redshift code.



Figure \insref.

Figure REF shows the bias (accuracy) and scatter (precision)
