

\section{Identifying high-$z$ galaxies with Euclid}

With one visible and 3 near-infrared bands \euclid\ provides the ability to robustly identify galaxies at high-redshift $z>6$ via the Lyman-$\alpha$/limit break. To demonstrate \euclid's capabilities we run a simple simulation in which we generate synthetic photometry assuming a simple $\beta$ model. In this model the underlying rest-frame ultraviolet spectrum is described by a simple power-law: $L_{\nu}\propto \lambda^{\beta+2}$ with ISM/IGM abosorption described by MADAU96.

We then add noise assuming the \euclid\ point-source depths and use the {\sc eazy} \insref\ photometric redshift code.



\begin{figure}
	\centering
	\includegraphics[width=0.45\textwidth]{./figures/beta/beta_dz.pdf}
	\caption{The difference between the median photometric redshift $z_{\rm EAZY}$ and the input redshift $z$ ($\Delta z = z_{\rm EAZY} - z$) as a function of input redshift. The solid line shows the median value of $\Delta z$ while the shaded region shows the central 68\% range. Input models assume a uniform distribution of UV continuum slopes $\beta\in[-3,0]$ and luminosities. Only galaxies with ${\rm S/N(H)}=5-20$ are included in the figure.\label{fig:beta_dz}}
\end{figure}

\begin{figure}
	\centering
	\includegraphics[width=0.45\textwidth]{./figures/beta/beta_bias.pdf}
  \includegraphics[width=0.45\textwidth]{./figures/beta/beta_scatter.pdf}
	\caption{The measured photometric redshift bias (top) and scatter (bottom) as a function of input redshift ($z$) and UV continuum slope ($\beta$). \label{fig:beta_bias_scatter}}
\end{figure}

