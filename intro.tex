\section{Introduction}\label{sec:intro}
The past few decades have seen tremendous growth in the understanding of galaxy formation and evolution in the first billion years of the Universe after the Big Bang. The first stars and galaxies formed within the first few million years after the big bang, were the first sources of ionising photons in the Universe, ushering in the Epoch of Reionisation (EoR) by ionising hydrogen \citep[\eg][]{wilkins2011,Bouwens2012b,Robertson2013,Robertson2015}. 


Thanks chiefly to the efforts of the \textit{Hubble Space Telescope} \citep[\hst\,, \eg][]{Beckwith_2006,Bouwens2008,Labbe2010,Robertson2010,Bouwens2014a,McLeod2015,Bowler2017,Kawamata_2018} and the \textit{Visible and Infrared Survey Telescope for Astronomy} \citep[\textit{VISTA}, \eg][]{Bowler2014, Stefanon2019,Bowler2020} more than a thousand galaxies have now been identified at $z>5$ with a handful of candidates even identified at $z>10$ \citep[\eg][]{Oesch_2016,Bouwens_2019}. These efforts have also been complemented by \textit{Spitzer} providing rest-frame optical photometry \citep[\eg][]{Ashby_2013,Roberts_Borsani_2016,Bridge_2019} and the \textit{Atacama Large Millimeter/submillimeter Array}  \citep[\textit{ALMA}, \eg][]{Smit2018,Carniani2018,Hashimoto2019} providing rest-frame far-IR and sub-mm photometry and spectroscopy.

With upcoming facilities like \textit{James Webb Space Telescope}, \euclid\,, \textit{Nancy Grace Roman Space Telescope} that can comprehensively study galaxies in the EoR, it is timely to model and predict the properties of these high redshift systems. The \textit{Webb Telescope} will be able to provide better sensitivity and spatial resolution in the near and mid-infrared, providing rest-frame UV-optical imaging and spectroscopy. \euclid\, and \rst\, can do deep and wide surveys adding better statistics to the bright end. The combine efforts can thus provide effective constraints on the bright and rare galaxies in the early Universe. This would further be the test beds for our understanding of the theory of galaxy formation and evolution giving insights into the physical processes that shape the star formation history, morphology and the impact of environment on the first galaxies.

One of the quantities in the EoR where we have extensive observational constraints is the galaxy UV luminosity function, measuring the comoving number density of galaxies as a function of their luminosity across different redshifts. There have been numerous studies done to quantify this value and provide insight into this population \citep[\eg][]{Bouwens_2015a,McLeod2015,Finkelstein2015,Livermore2017,Atek2018,Stefanon2019,Bowler2020}. Another exciting area which is currently being probed are line luminosities and their equivalent widths. Lyman alpha has been primarily used for spectroscopic confirmation of high-redshift galaxies, but becomes increasingly weak at higher-redshift due to increasing neutral fraction in the inter-galactic medium (IGM). Rest-frame far infrared lines are also useful probe of galaxies in the EoR. ALMA has also found mixed success in detecting the brightest of the far-infrared lines like [CII] and [OIII] even in some of the highest redshift galaxies \citep[\eg][]{Hashimoto2018,Harikane2019}. Many of the emission lines in the optical, arises from HII regions rather than from photo-dissociation regions (PDRs). This makes their modelling easier compared to the latter. With most of the current existing constraints in the EoR coming from luminosity functions in the UV; this will change with the lauch of \jwst\, whose onboard instruments will provide access to many of the strong emission lines in the EoR. 

Complementary to this many theoretical works on simulations of galaxy evolution have been used to study the population of galaxies and their properties in the EoR. For testing the validity of these theoretical models, they need to make predictions on the various observables. There are various intrinsic physical properties of galaxies like stellar mass, star formation rate, that are available directly from simulations which can be compared to that of observed galaxies. These all involve some modelling assumptions based on the star formation history or metallicity of the observed galaxies, which are hard to derive with limited available data on the galaxy at these high redshifts. Another approach is to make predictions from simulations to compare that to galaxy observables that suffer from comparatively less modelling biases and thus giving us insights into the physical processes that takes place in these galaxies.  

Semi-Anlytical Models (SAMs), which run on halo merger trees extracted from dark matter only simulations or Extended Press-Schechter methods have been widely used and very successful in the study of galaxy formation and evolution \citep[\eg][]{henriques2015,Somerville2015,Rodrigues2017}. A number of these studies have been used to make predictions on the observables in the EoR \citep[\eg][]{Clay2015,Poole2016,Lacey2016,Yung2019,Hutter2020}. They are powerful tools that can be applied to large cosmological volumes to extend the range of the probed distribution functions or observables due to their shorter computation times, with each generation of SAMs incorporating more detailed physical models in them; however they rely on analytical recipes, thus do not self-consistently evolve various interactions, requiring additional steps and approximations to retrieve observables.

Hydrodynamical simulations of galaxy formation that self-consistently trace the evolution of dark matter, gas, stars and black holes are another tool to study the evolution of galaxy properties. 
Many state of the art periodic cosmological volumes \citep[\eg][]{vogelsberger_introducing_2014,Dubois2014,Khandai2015,schaye_eagle_2015,Feng2016,dave_mufasa:_2016-1,Pillepich2018,dave_simba:_2019} have been undertaken and they have been successful in reproducing many of the observables, but only few of these simulation suites have the capability to provide the large scale power to replicate these observations. The enormous computational time to run these large periodic boxes have been a major roadblock from exploring large dynamic ranges with better resolution. There have been high resolution zoom simulations that have probed the EoR \citep[\eg][]{Ceverino2017,Ma2018} but not necessarily extended the dynamic range that will be probed by the next generation surveys. For the purpose of studying this epoch in the Universe, we have run a suite of zoom-in simulations probing a range of overdensities, termed First Light and Reionisation Epoch Simulations, \textsc{Flares}; see \cite{lovell2020}, hereafter \citetalias{lovell2020}. 

In this article we use the \textsc{Flares} suite of resimulations to predict the photometric properties of the galaxies in the EoR which will be accessible to the upcoming \textit{Webb}, \euclid\,,  \textit{Roman} telescopes, thus providing insights into current galaxy modelling physics of the high redshift Universe. We begin by briefly introducing the simulation suite in Section~\S\ref{sec:simulations} and our modelling of galaxy observables in Section~\S\ref{sec:simulations.SED} and \S\ref{sec:modelling.dust}. In Section~\S\ref{sec:PhotProp} we focus on the derived photometric properties of the simulated galaxies like the UV luminosity function and nebular line emission properties, and present our conclusions in Section~\S\ref{sec:conc}. We assume a Planck year 1 cosmology \citep[$\Omega_{m}$ = 0.307, $\Omega_{\Lambda}$ = 0.693, h = 0.6777;][]{planck_collaboration_2014}. 